% Created 2014-08-05 Tue 11:09
\documentclass[11pt]{article}
\usepackage[utf8]{inputenc}
\usepackage[T1]{fontenc}
\usepackage{fixltx2e}
\usepackage{graphicx}
\usepackage{longtable}
\usepackage{float}
\usepackage{wrapfig}
\usepackage{rotating}
\usepackage[normalem]{ulem}
\usepackage{amsmath}
\usepackage{textcomp}
\usepackage{marvosym}
\usepackage{wasysym}
\usepackage{amssymb}
\usepackage{hyperref}
\tolerance=1000
\author{SunWenxiang}
\date{\today}
\title{Oh My Emacs}
\hypersetup{
 pdfkeywords={},
  pdfsubject={},
  pdfcreator={Emacs 24.3.92.1 (Org mode )}}
\begin{document}

\maketitle
\setcounter{tocdepth}{2}
\tableofcontents

\begin{quote}
Emacs outshines all other editing software in approximately the same
way that the noonday sun does the stars. It is not just bigger and
brighter; it simply makes everything else vanish.

-- Neal Stephenson, "In the Beginning was the Command Line"
\end{quote}

\section*{Prerequisites}
\label{sec-1}
This project was born for emacs 24+, emacs 23 is not officially supported and
there's no plan to support it. One of the builtin package of oh-my-emacs, \href{https://github.com/emacs-helm/helm}{helm},
required Emacs version 24.3+, and oh-my-emacs will work best with recent
version emacs.

The following software is really basic that almost every el-get package will
need one of them.

\begin{table}[htb]
\caption{\label{ome-prerequisites}Prerequisites for all oh-my-emacs module.}
\centering
\begin{tabular}{lllllll}
 & Windows & Ubuntu/Debian/Mint & ArchLinux & Fedora & Mac OS X & Mandatory?\\
\hline
\href{https://www.gnu.org/software/emacs/}{emacs 24.3+} &  & emacs24(emacs-snapshot) &  &  &  & Yes\\
\href{http://savannah.nongnu.org/projects/cvs}{cvs} &  & cvs &  & cvs &  & Yes\\
\href{http://subversion.apache.org/}{subversion} &  & subversion &  & subversion &  & Yes\\
\href{http://mercurial.selenic.com/}{mercurial} &  & mercurial &  & mercurial &  & Yes\\
\href{http://git-scm.com/}{git} &  & git &  & git &  & Yes\\
\href{http://www.gnu.org/software/make/}{make} &  & make &  & make &  & Yes\\
\href{http://www.gnu.org/software/texinfo/}{texinfo} &  & texinfo &  & texinfo &  & Yes\\
\href{http://gcc.gnu.org/}{gcc} &  & gcc &  & gcc &  & Yes\\
\href{http://www.gnu.org/software/automake/}{automake} &  & automake &  & automake &  & Yes\\
\href{http://www.gnu.org/software/autoconf/}{autoconf} &  & autoconf &  & autoconf &  & Yes\\
\end{tabular}
\end{table}

\section*{Introduction}
\label{sec-2}
This dotemacs project aims to provide an out-of-the-box emacs configuration
for writing and programming. It is intended for both beginners and
intermediate users. The name of this project is inspired by \href{https://github.com/robbyrussell/oh-my-zsh}{oh-my-zsh}, which is
another out-of-the-box configuration for \href{http://www.zsh.org/}{zsh}.

The main features of oh-my-emacs are
\begin{itemize}
\item better default settings
\item focus on writing and programming, let emacs do what it really good at
\item "literate" customization embedded in Org-mode files
\item use el-get to manage packages, which is portable, and version control
friendly
\item self-adapting, enable features only when the system meet the requirements
\end{itemize}

The latest version is at \url{https://github.com/xiaohanyu/oh-my-emacs}.

\subsection*{Learning Emacs}
\label{sec-2-1}
\begin{quote}
People talk about getting used to a new editor, but over time, it is precisely
the opposite that should happen --- the editor should get used to us.

-- Vivek Haldar
\end{quote}

This project won't teach you anything about how to use emacs or why use
emacs. However, I'll give you some useful resources if you're new to emacs.

The first thing you should refer to is the emacs builtin tutorial, type \texttt{C-h
t} to get it within a running emacs.

Next, you should refer to some general books to get a full view of emacs,
just search \url{http://www.amazon.com}.

Now you should be a intermediate emacs user(or lover, I hope), the final step
is emacs lisp, which is not that easy as it looks like, since it is not easy to
get a full understanding of the so-called lisp. But you can still do something
useful even if you didn't know lisp quite a lot. The builtin \href{https://www.gnu.org/software/emacs/manual/html_node/eintr/}{An Introduction to
Programming in Emacs Lisp} is a really good entry point to emacs lisp
programming. You can refer to this tutorial by typing \texttt{C-h i m Emacs Lisp
Intro}.

Finally, if you want to do some big jobs in emacs lisp, then there's a long
adventure waiting for you. The builtin \href{http://www.gnu.org/software/emacs/manual/}{GNU Emacs Manual} and \href{http://www.gnu.org/software/emacs/manual/html_node/elisp/}{GNU Emacs Lisp
Reference Manual} is your best friends. Besides these two large books, there're
various good web sites, such as
\begin{description}
\item[{\href{http://www.emacswiki.org/}{emacswiki}}] the largest unofficial emacs lisp repository
\item[{\href{http://wikemacs.org/index.php/Main_Page}{wikiemacs}}] a saner and modern wiki for emacs
\item[{\href{http://www.masteringemacs.org/}{masteringemacs}}] a blog about mastering the world's best text editor
\item[{\href{http://emacs-fu.blogspot.com/}{emacs-fu}}] useful tricks for emacs
\item[{\href{http://alexott.net/en/emacs/}{Alexott's Blog}}] some really great articles about emacs
\item[{\href{http://alexott.net/en/emacs/}{emacser}}] an excellent and comprehensive emacs blog written by Chinese
users.
\item[{\href{http://whattheemacsd.com/}{whattheemacsd}}] contains some really great tips.
\end{description}

Wow, what a long journey, if you're tired, you can \href{http://emacsrocks.com/}{rock with emacs}.

\subsection*{Installation}
\label{sec-2-2}
\begin{enumerate}
\item Install Emacs version 24 or greater.  Use your package manager if
you have one and it has an install candidate for Emacs 24,
otherwise install it directly \href{http://savannah.gnu.org/projects/emacs/}{from source}, or Mac binaries may be
downloaded from the \emph{nightlies} section of
\url{http://emacsformacosx.com/builds}
\item Checkout a version of the oh-my-emacs using git -- if you're new to
git, checkout this \href{http://www.kernel.org/pub/software/scm/git/docs/gittutorial.html}{git-tutorial}.
\end{enumerate}
\begin{verbatim}
git clone http://github.com/xiaohanyu/oh-my-emacs.git
\end{verbatim}
\begin{enumerate}
\item Move the resulting directory to \texttt{\textasciitilde{}/.emacs.d} \footnote{If you already have a directory at \texttt{\textasciitilde{}/.emacs.d} move it out of the way and
put this there instead. Or you can make a symbolic link.}
\item Launch Emacs!
\end{enumerate}


\subsection*{Package Management}
\label{sec-2-3}
\begin{quote}
Let's face it -- although a vanilla Emacs installation is quite powerful almost
nobody is using Emacs without a pile of add-ons. And managing those add-ons is
quite frankly a pain in the ass. Traditional options included installing Emacs
add-ons via the operating system's package manager (if available), downloading
.el files from various locations (everybody hates packages distributed only on
Emacs Wiki with no canonical version control repo) and simply sticking them on
the load-path, etc. It's more than obvious that such solutions are less than
ideal.

For instance if you're installing Emacs add-ons via a package manager and you
have to change OSes (or machines) you're mostly fucked. On the other hand
piling files manually in .emacs.d is equal to hell in the version and
dependency tracking department. There has to be a better way, right? Wouldn't
it be nice if Emacs had its own package manager similar to the likes of
homebrew, apt or yum?

Emacs 24 finally introduces such a tool -- its name is package.el (very
original, right?) and it's promise is to make your lives a bit easier. Does it
manage to deliver on that promise? We'll see that in a bit\ldots{}

-- \href{http://batsov.com/articles/2012/02/19/package-management-in-emacs-the-good-the-bad-and-the-ugly/}{Package Management in Emacs: The Good, the Bad and the Ugly}
\end{quote}

Ah, a long long quote, really vivid and visual description of the state of
emacs package management system. \href{https://github.com/xiaohanyu/oh-my-emacs}{oh-my-emacs} adopt \href{https://github.com/dimitri/el-get}{el-get}, while other popular
dotemacs project adopt the builtin \texttt{package.el}. You may wonder why, IMHO,
\texttt{package.el} is far from ideal, although it is builtin with emacs, however,
el-get is the practical solution, at least for now. In a word, what el-get
to emacs is what \texttt{apt-get} to debian/ubuntu.

So the very first thing oh-my-emacs do to your emacs is to install \texttt{el-get},
this is done by putting the following code snippet in \texttt{\$HOME/.emacs.d/init.el}.

\begin{verbatim}
(add-to-list 'load-path "~/.emacs.d/el-get/el-get")

(unless (require 'el-get nil 'noerror)
  (with-current-buffer
      (url-retrieve-synchronously
       "https://raw.github.com/dimitri/el-get/master/el-get-install.el")
    (let (el-get-master-branch
          ;; do not build recipes from emacswiki due to poor quality and
          ;; documentation
          el-get-install-skip-emacswiki-recipes)
      (goto-char (point-max))
      (eval-print-last-sexp)))
  ;; build melpa packages for el-get
  (el-get-install 'package)
  (setq package-archives '(("gnu" . "http://elpa.gnu.org/packages/")
                           ("melpa" . "http://melpa.milkbox.net/packages/")))
  (el-get-elpa-build-local-recipes))

(add-to-list 'el-get-recipe-path "~/.emacs.d/ome-el-get-recipes")

;; (el-get 'sync)
\end{verbatim}

Note that by default, el-get support multiple method to install emacs packages
from various sources, such as elpa, emacswiki, etc. However, IMHO, most of the
emacswiki packages have a poor quality and documentation. So I disable it by
default.

By default, el-get bundles with about 1000 packages. Combined with elpa,
there're about 3000 packages. However, many of them are outdated, unmaintained,
lack of documentation. For elpa package archive, there are multiple choices,
namely the official GNU \href{http://elpa.gnu.org/}{elpa}, \href{http://marmalade-repo.org/}{marmalade} and \href{http://melpa.milkbox.net/}{melpa}, etc. \href{http://elpa.gnu.org/}{elpa} is just too stable
which update just too little, and packages in \href{http://marmalade-repo.org/}{marmalade} is a little outdated
since it requires authors of packages to upload their package each time a
new version is released. \href{http://melpa.milkbox.net/}{melpa}, similar to el-get, however, it builds packages
directly from package's source repository. To tell the truth, melpa is really
quite a good alternative to el-get. Compared to el-get, there's no need to
install \texttt{git}, \texttt{cvs}, \texttt{svn}, etc. You have \texttt{package.el}, then you can use
melpa. However, if you want better integration between your emacs packages with
your system, just use el-get. For technical details, check \href{https://github.com/milkypostman/melpa/issues/742}{here}.

Oh-my-emacs adopt el-get, while enabled only the official elpa and melpa repo
for \texttt{package.el}, it also \texttt{(el-get-elpa-build-local-recipes)} when el-get is
first installed, so you can install melpa package directly via el-get.

And in the following journey, you may find the great power \texttt{el-get} provides
for you -- which makes it possible to store just dotemacs configuration, no
need to things like \texttt{git submodule}. Thank you, \href{http://tapoueh.org/}{Dimitri Fontaine}.

\subsection*{Cl-lib}
\label{sec-2-4}
\href{http://www.gnu.org/software/emacs/news/NEWS.24.3}{Emacs 24.3} renamed the Common Lisp emulation package from \texttt{cl} to \texttt{cl-lib},
which causes some madness between different Emacs versions. The following hack
code has been tested on Emacs 24.2.1 and Emacs 24.3.50. Any bug report will be
appreciated.

\begin{verbatim}
(when (version< emacs-version "24.3")
  (el-get 'sync '(cl-lib))
  (add-to-list 'load-path "~/.emacs.d/el-get/cl-lib"))
\end{verbatim}

\subsection*{Structure}
\label{sec-2-5}
The \texttt{init.el} file is where everything begins. It is loaded
automatically by Emacs on startup, its sole purpose is to load the
elisp code embedded in this file.

\section*{Implementation}
\label{sec-3}

\subsection*{oh-my-emacs basics}
\label{sec-3-1}
\begin{itemize}
\item Functions for loading other parts of oh-my-emacs.
\end{itemize}

\begin{verbatim}
(defvar ome-packages nil
  "This package contains all packages loaded by oh-my-emacs.

Acutally, this variable is an alist, whose element is also a
list. The car of the element is an oh-my-emacs module name, while
the cdr of the element is a list of el-get packages loaded in particular
oh-my-emacs module.")

(defun ome-load (module &rest header-or-tags)
  "Load configuration from other ome-*.org files.
If the optional argument is the id of a subtree then only
configuration from within that subtree will be loaded.  If it is
not an id then it will be interpreted as a tag, and only subtrees
marked with the given tag will be loaded.

For example, to load all of ome-lisp.org simply add (ome-load
\"ome-lisp\") to your configuration.

To load only the 'window-system' config from ome-miscs.org
add (ome-load \"ome-miscs.org\" \"window-system\") to your
configuration.

The good news is, you can load multiple parts config from one
single file by simply (ome-load \"ome-module.org\" \"part1\"
\"part2\")."
  (let ((module-name (file-name-base module))
        (file (expand-file-name (if (string-match "ome-.+\.org" module)
                                    module
                                  (format "ome-%s.org" module))
                                ome-dir)))
    ;; ensure el-get-sources is empty before loading "ome-.+\.org" files
    (setq el-get-sources nil)
    ;; enable git shallow clone to save time and bandwidth
    (setq el-get-git-shallow-clone t)

    (if header-or-tags
        (dolist (header-or-tag header-or-tags)
          (let* ((base (file-name-nondirectory file))
                 (dir  (file-name-directory file))
                 (partial-file (expand-file-name
                                (concat "." (file-name-sans-extension base)
                                        ".part." header-or-tag ".org")
                                dir)))
            (unless (file-exists-p partial-file)
              (with-temp-file partial-file
                (insert
                 (with-temp-buffer
                   (insert-file-contents file)
                   (save-excursion
                     (condition-case nil ;; collect as a header
                         (progn
                           (org-link-search (concat "#" header-or-tag))
                           (org-narrow-to-subtree)
                           (buffer-string))
                       (error ;; collect all entries with as tags
                        (let (body)
                          (org-map-entries
                           (lambda ()
                             (save-restriction
                               (org-narrow-to-subtree)
                               (setq body (concat body "\n" (buffer-string)))))
                           header-or-tag)
                          body))))))))
            (org-babel-load-file partial-file)))
      (org-babel-load-file file))

    (el-get 'sync (mapcar 'el-get-source-name el-get-sources))
    (setq ome-module-packages nil)
    (mapcar (lambda (el-get-package)
              (add-to-list 'ome-module-packages
                           (el-get-source-name el-get-package)))
            el-get-sources)
    (add-to-list 'ome-packages
                 (cons module-name ome-module-packages))))
\end{verbatim}

\begin{itemize}
\item Functions for installing el-get packages
\end{itemize}

\begin{verbatim}
(defun ome-install (el-get-package)
  "Add EL-GET-PACKAGE to `el-get-sources'.

This package will be installed when `ome-load'. Users can make
his own customization by providing a \"ome-package-name-setup\"
function."
  (let ((ome-package-setup-func
         (intern
          (concat "ome-"
                  (el-get-as-string el-get-package)
                  "-setup"))))
    (if (fboundp ome-package-setup-func)
        (add-to-list 'el-get-sources
                     `(:name ,el-get-package
                             :after (progn
                                      (,ome-package-setup-func))))
      (add-to-list 'el-get-sources
                   `(:name ,el-get-package)))))
\end{verbatim}

\begin{itemize}
\item Functions for getting information about el-get packages installed by
oh-my-emacs.
\end{itemize}
\begin{verbatim}
(defun ome-try-get-package-website (package)
  "el-get's package recipe has multiple type, some contains
a :website, some contains just a :url, while some github package
just contains a :pkgname. This function try to get a proper
website link for an el-get package."
  (let ((package-def (el-get-package-def package)))
    (or (plist-get package-def :website)
        (and (eq (plist-get package-def :type) 'github)
             (concat "https://github.com/"
                     (plist-get package-def :pkgname)))
        (plist-get package-def :url))))

(defun ome-try-get-package-description (package)
  "This function try to get a proper description for an el-get
package from its recipe. Note that some package's description has
multiple lines, so we need to join them together for better
auto-fill."
  (let ((package-def (el-get-package-def package)))
    (replace-regexp-in-string "\\(\n\\|\\ \\)+" " "
                              (plist-get package-def :description))))

(defun ome-package-list-to-org-table ()
  (interactive)
  (setq ome-packages (sort ome-packages
                           #'(lambda (x y) (string< (car x) (car y)))))
  (let ((org-table-line-template "|%s|[[%s][%s]]|%s|\n"))
    (with-temp-buffer
      (insert "| Module | Package | Description |\n")
      (insert "|--------+---------+-------------|\n")
      (insert "")
      (dolist (module-packages ome-packages)
        (setq package-index 0)
        (dolist (package (cdr module-packages))
          (insert (format org-table-line-template
                          (if (= package-index 0)
                              (car module-packages)
                            "")
                          (ome-try-get-package-website package)
                          package
                          (ome-try-get-package-description package)))
          (incf package-index)))
      (buffer-string))))
\end{verbatim}

\subsection*{oh-my-emacs core functions}
\label{sec-3-2}
\begin{verbatim}
(add-to-list 'el-get-sources
             '(:name cl-lib))

(defun ome-start-or-switch-to (function buffer-name)
  "Invoke FUNCTION if there is no buffer with BUFFER-NAME.
  Otherwise switch to the buffer named BUFFER-NAME.  Don't clobber
  the current buffer."
  (if (not (get-buffer buffer-name))
      (progn
        (split-window-sensibly (selected-window))
        (other-window 1)
        (funcall function))
    (switch-to-buffer-other-window buffer-name)))
\end{verbatim}

\subsection*{Writing Portable Emacs Lisp Configurations}
\label{sec-3-3}
\begin{itemize}
\item Use \texttt{executable-find} to set path to external tools instead of by writing the
tool full path by hand.
\item Check necessary external tools before require a package if a package do need
that tool.
\item Try your best to avoid binding keys to function keys since some function keys
are easy be conflicted with system keys, and some keyboard do not have
function keys at all, such as HHKB.
\item Narrow you variable definition scope, for example, use \texttt{define-key} with a
mode map instead of \texttt{global-set-key}, and you'd better set mode related keys
within the mode hook.
\item Try to use system tools to get necessary configuration, for example, you can
use \texttt{pkg-config} to get necessary header file paths for a particular lib,
this is useful for settings with \texttt{auto-complete-clang}.
\item Emacs cannot do all things, it do need some external tools to facilitate its
jobs. For example, \texttt{semantic} from \texttt{cedet} is a bad idea, it's bloated, slow
and buggy. Consider using clang/jedi for completion. If there is a better
external tools, do not hesitate, just use it. Do not reinvent the same wheel
in emacs lisp, all you need to do is integrate that tool to emacs with an emacs
interface.
\end{itemize}

\section*{Load settings}
\label{sec-4}

Files in oh-my-emacs \texttt{core} will be loaded by default:
\begin{itemize}
\item \href{core/ome-auto-mode.org}{ome-auto-mode.org}: small and convenient settings for some minor mode package.
\item \href{core/ome-basic.org}{ome-basic.org}: basic settings, such as tab, space, auto-fill-mode, flyspell,
\item \href{core/ome-completion.org}{ome-completion.org}: main settings for completion support.
\item \href{core/ome-gui.org}{ome-gui.org}: settings for color theme, font, modeline, scrollbar, etc.
etc.
\item \href{core/ome-keybindings.org}{ome-keybindings.org}: oh-my-emacs keybindings.
\item \href{core/ome-miscs.org}{ome-miscs.org}: misc settings such as \texttt{magit}, \texttt{smartparens}, \texttt{projectile},
etc.
\item \href{core/ome-org.org}{ome-org.org}: some refinement for org-mode.
\item \href{core/ome-writing.org}{ome-writing.org}: for writers, bloggers, only contains \texttt{markdown-mode} for
now.
\item \href{core/ome-advanced.org}{ome-advanced.org}: contains \texttt{evil}, \texttt{ace-jump-mode}, \texttt{ack-and-a-half}, etc.
\end{itemize}

Files in oh-my-emacs \texttt{modules} is optional, you can load necessary one when
you really need it.
\begin{itemize}
\item \href{modules/ome-cc.org}{ome-cc.org}: main settings for \texttt{cc-mode}.
\item \href{modules/ome-common-lisp.org}{ome-common-lisp.org}: settings for Common Lisp, mainly \href{http://common-lisp.net/project/slime/}{SLIME}.
\item \href{modules/ome-clojure.org}{ome-clojure.org}: settings for Clojure.
\item \href{modules/ome-emacs-lisp.org}{ome-emacs-lisp.org}: settings for emacs-lisp.
\item \href{modules/ome-java.org}{ome-java.org}: settings for Java, using \href{https://github.com/senny/emacs-eclim}{emacs-eclim}.
\item \href{modules/ome-javascript.org}{ome-javascript.org}: settings for JavaScript, not perfect now.
\item \href{modules/ome-python.org}{ome-python.org}: python settings, life is short, \texttt{elpy} is amazing.
\item \href{modules/ome-ruby.org}{ome-ruby.org}: Ruby support including \texttt{inf-ruby} and by \texttt{smartparens-ruby}.
\item \href{modules/ome-scheme.org}{ome-scheme.org}: settings for Scheme.
\item \href{modules/ome-ocaml.org}{ome-ocaml.org}: minimal settings for OCaml.
\item \href{modules/ome-tex.org}{ome-tex.org}: AUCTeX and CDLaTeX for \TeX{} editing.
\item \href{modules/ome-web.org}{ome-web.org}: settings for HTML/CSS web development.
\item \href{modules/ome-experimental.org}{ome-experimental.org}: some experimental packages, such as sppedbar, minimap,
etc. Not stable yet now.
\end{itemize}

\begin{verbatim}
(ome-load "core/ome-auto-mode.org")
(ome-load "core/ome-basic.org")
(ome-load "core/ome-completion.org")
(ome-load "core/ome-gui.org")
(ome-load "core/ome-keybindings.org")
(ome-load "core/ome-miscs.org")
(ome-load "core/ome-org.org")
(ome-load "core/ome-writing.org")
(ome-load "core/ome-advanced.org")
(ome-load "modules/ome-cc.org")
(ome-load "modules/ome-java.org")
(ome-load "modules/ome-emacs-lisp.org")
(ome-load "modules/ome-common-lisp.org")
(ome-load "modules/ome-clojure.org")
(ome-load "modules/ome-scheme.org")
(ome-load "modules/ome-python.org")
(ome-load "modules/ome-ruby.org")
(ome-load "modules/ome-ocaml.org")
(ome-load "modules/ome-tex.org")
(ome-load "modules/ome-web.org")
(ome-load "modules/ome-experimental.org" "smooth-scrolling" "sublimity")
\end{verbatim}

\subsection*{Load User/System Specific Files}
\label{sec-4-1}
\subsubsection*{How to disable an el-get package?}
\label{sec-4-1-1}

To modify ome itself, you need to know something about \href{http://orgmode.org/worg/org-contrib/babel/intro.html#literate-programming}{org-babel}. In general,
all ome packages are loaded in a org-mode's code block, the general format of
which is:

\begin{verse}
\#+NAME: <name> \\
\#+BEGIN\_SRC <language> <switches> <header arguments> \\
\hspace*{7em}<body> \\
\#+END\_SRC \\
\end{verse}

The \href{http://orgmode.org/manual/Header-arguments.html#Header-arguments}{header arguments} can control things like whether code is executed when
\texttt{org-babel-load-file}, or exported when \texttt{org-export-xxxx}. If you want to
disable a code block in \texttt{org-babel-load-file}, you should use the \href{http://orgmode.org/manual/tangle.html#tangle}{:tangle}
header argument by settings it to \texttt{:tangle no}. Thus this code block will not
be tangled by org-babel and ome won't execute this code block when starts.

For example, here's a general code block for package-abc:

\begin{verse}
\#+BEGIN\_SRC emacs-lisp \\
(defun ome-package-abc-setup () \\
\hspace*{2em};; some custom code for package-abc \\
\hspace*{2em}) \\
\vspace*{1em}
(ome-install 'package-abc) \\
\#+END\_SRC \\
\end{verse}

If you want to disable package-abc, you just add a \texttt{:tangle no} to this code
block:

\begin{verse}
\#+BEGIN\_SRC emacs-lisp :tangle no \\
(defun ome-package-abc-setup () \\
\hspace*{2em};; some custom code for package-abc \\
\hspace*{2em}) \\
\vspace*{1em}
(ome-install 'package-abc) \\
\#+END\_SRC \\
\end{verse}

Although you disabled package-abs'c code block, package-abs still keep
installed in your emacs. Sometimes, you even want to disable this package from
the bottom. Then you should first remove package-abc via \texttt{el-get-remove
package-abc}, and add \texttt{:tangle no} to that code block.

\subsubsection*{General custom code}
\label{sec-4-1-2}

Ome push some small code snippets to \texttt{custom.el}.

\begin{verbatim}
(setq custom-file (concat ome-dir "custom.el"))
(load custom-file 'noerror)
\end{verbatim}

\subsubsection*{System/User specific customizations}
\label{sec-4-1-3}
You can keep system- or user-specific customizations here in either raw
emacs-lisp files or as embedded elisp in org-mode files (as done in this
document).

You can keep elisp source in the \texttt{src} directory. Packages loaded from here
will override those installed by ELPA.  This is useful if you want to track the
development versions of a project, or if a project is not in elpa.

For example, if your username is \texttt{testuser}, then you can put a \texttt{testuser.el}
file in your \texttt{\$HOME/.emacs.d}, and then put anything you like in
\texttt{testuser.el}. You may wonder what's the difference between ome's \texttt{custom.el}
and \texttt{testuser.el}. Actually, \texttt{custom.el} is ome's own code snippets, while
\texttt{testuser.el} is your own code snippet and no need to \texttt{git add testuser.el}. So
you can keep sync with upstream while keep your own customizations.

After we've loaded all the oh-my-emacs defaults, lets load the user's stuff.

\begin{verbatim}
(flet ((sk-load (base)
         (let* ((path          (expand-file-name base ome-dir))
                (literate      (concat path ".org"))
                (encrypted-org (concat path ".org.gpg"))
                (plain         (concat path ".el"))
                (encrypted-el  (concat path ".el.gpg")))
           (cond
            ((file-exists-p encrypted-org) (org-babel-load-file encrypted-org))
            ((file-exists-p encrypted-el)  (load encrypted-el))
            ((file-exists-p literate)      (org-babel-load-file literate))
            ((file-exists-p plain)         (load plain)))))
       (remove-extension (name)
         (string-match "\\(.*?\\)\.\\(org\\(\\.el\\)?\\|el\\)\\(\\.gpg\\)?$" name)
         (match-string 1 name)))
  (let ((elisp-dir (expand-file-name "src" ome-dir))
        (user-dir (expand-file-name user-login-name ome-dir)))
    ;; add the src directory to the load path
    (add-to-list 'load-path elisp-dir)
    ;; load specific files
    (when (file-exists-p elisp-dir)
      (let ((default-directory elisp-dir))
        (normal-top-level-add-subdirs-to-load-path)))
    ;; load system-specific config
    (sk-load system-name)
    ;; load user-specific config
    (sk-load user-login-name)
    ;; load any files in the user's directory
    (when (file-exists-p user-dir)
      (add-to-list 'load-path user-dir)
      (mapc #'sk-load
            (remove-duplicates
             (mapcar #'remove-extension
                     (directory-files user-dir t ".*\.\\(org\\|el\\)\\(\\.gpg\\)?$"))
             :test #'string=)))))
\end{verbatim}

\section*{Todo}
\label{sec-5}

\subsection*{Blueprint}
\label{sec-5-1}
\begin{itemize}
\item Add functions to rollback el-get. For example, users may have different
environments, which may result in a failure of one oh-my-emacs module A. Thus
users may want to remove all the packages installed by module A. So maybe I
need a function \texttt{(ome-unload "ome-module-A.org")}, which just remove all
oh-my-emacs by calling \texttt{el-get-remove}, and rollback the user's .emacs.d to
a previous workable state(hope).
\end{itemize}

\subsection*{Boot up speed optimization}
\label{sec-5-2}
\begin{itemize}
\item Use \texttt{eval-after-load} when possible to improve the boot up speed.
\item \texttt{autoload} for lazy loading?
\end{itemize}

\subsection*{El-get}
\label{sec-5-3}
\begin{itemize}
\item Investigate whether or not \texttt{el-get-remove} will remove all packages that
depends on the removed package, if not, how to solve this problem?
\item Investigate more on \texttt{el-get} to know how to reinitialize a package without
need to \texttt{el-get-remove} then \texttt{el-get-install} if just \texttt{:after} changes a
little. This is really a big problem.
\item Seems that you can write \texttt{depends} in ome \texttt{el-get-sources}, why?
\item When you \texttt{el-get-update}, \texttt{el-get} will first \texttt{el-get-remove} then
\texttt{el-get-install}, any way to just do things like \texttt{git pull} instead of \texttt{git
  clone} from the beginning?
\end{itemize}
% Emacs 24.3.92.1 (Org mode )
\end{document}
